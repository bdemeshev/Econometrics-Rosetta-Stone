\documentclass[]{book}
\usepackage{lmodern}
\usepackage{amssymb,amsmath}
\usepackage{ifxetex,ifluatex}
\usepackage{fixltx2e} % provides \textsubscript
\ifnum 0\ifxetex 1\fi\ifluatex 1\fi=0 % if pdftex
  \usepackage[T1]{fontenc}
  \usepackage[utf8]{inputenc}
\else % if luatex or xelatex
  \ifxetex
    \usepackage{mathspec}
  \else
    \usepackage{fontspec}
  \fi
  \defaultfontfeatures{Ligatures=TeX,Scale=MatchLowercase}
\fi
% use upquote if available, for straight quotes in verbatim environments
\IfFileExists{upquote.sty}{\usepackage{upquote}}{}
% use microtype if available
\IfFileExists{microtype.sty}{%
\usepackage{microtype}
\UseMicrotypeSet[protrusion]{basicmath} % disable protrusion for tt fonts
}{}
\usepackage[margin=1in]{geometry}
\usepackage{hyperref}
\hypersetup{unicode=true,
            pdftitle={Розеттский камень},
            pdfauthor={Пуассон, фея и два мексиканских негодяя},
            pdfborder={0 0 0},
            breaklinks=true}
\urlstyle{same}  % don't use monospace font for urls
\usepackage{natbib}
\bibliographystyle{apalike}
\usepackage{color}
\usepackage{fancyvrb}
\newcommand{\VerbBar}{|}
\newcommand{\VERB}{\Verb[commandchars=\\\{\}]}
\DefineVerbatimEnvironment{Highlighting}{Verbatim}{commandchars=\\\{\}}
% Add ',fontsize=\small' for more characters per line
\usepackage{framed}
\definecolor{shadecolor}{RGB}{248,248,248}
\newenvironment{Shaded}{\begin{snugshade}}{\end{snugshade}}
\newcommand{\KeywordTok}[1]{\textcolor[rgb]{0.13,0.29,0.53}{\textbf{#1}}}
\newcommand{\DataTypeTok}[1]{\textcolor[rgb]{0.13,0.29,0.53}{#1}}
\newcommand{\DecValTok}[1]{\textcolor[rgb]{0.00,0.00,0.81}{#1}}
\newcommand{\BaseNTok}[1]{\textcolor[rgb]{0.00,0.00,0.81}{#1}}
\newcommand{\FloatTok}[1]{\textcolor[rgb]{0.00,0.00,0.81}{#1}}
\newcommand{\ConstantTok}[1]{\textcolor[rgb]{0.00,0.00,0.00}{#1}}
\newcommand{\CharTok}[1]{\textcolor[rgb]{0.31,0.60,0.02}{#1}}
\newcommand{\SpecialCharTok}[1]{\textcolor[rgb]{0.00,0.00,0.00}{#1}}
\newcommand{\StringTok}[1]{\textcolor[rgb]{0.31,0.60,0.02}{#1}}
\newcommand{\VerbatimStringTok}[1]{\textcolor[rgb]{0.31,0.60,0.02}{#1}}
\newcommand{\SpecialStringTok}[1]{\textcolor[rgb]{0.31,0.60,0.02}{#1}}
\newcommand{\ImportTok}[1]{#1}
\newcommand{\CommentTok}[1]{\textcolor[rgb]{0.56,0.35,0.01}{\textit{#1}}}
\newcommand{\DocumentationTok}[1]{\textcolor[rgb]{0.56,0.35,0.01}{\textbf{\textit{#1}}}}
\newcommand{\AnnotationTok}[1]{\textcolor[rgb]{0.56,0.35,0.01}{\textbf{\textit{#1}}}}
\newcommand{\CommentVarTok}[1]{\textcolor[rgb]{0.56,0.35,0.01}{\textbf{\textit{#1}}}}
\newcommand{\OtherTok}[1]{\textcolor[rgb]{0.56,0.35,0.01}{#1}}
\newcommand{\FunctionTok}[1]{\textcolor[rgb]{0.00,0.00,0.00}{#1}}
\newcommand{\VariableTok}[1]{\textcolor[rgb]{0.00,0.00,0.00}{#1}}
\newcommand{\ControlFlowTok}[1]{\textcolor[rgb]{0.13,0.29,0.53}{\textbf{#1}}}
\newcommand{\OperatorTok}[1]{\textcolor[rgb]{0.81,0.36,0.00}{\textbf{#1}}}
\newcommand{\BuiltInTok}[1]{#1}
\newcommand{\ExtensionTok}[1]{#1}
\newcommand{\PreprocessorTok}[1]{\textcolor[rgb]{0.56,0.35,0.01}{\textit{#1}}}
\newcommand{\AttributeTok}[1]{\textcolor[rgb]{0.77,0.63,0.00}{#1}}
\newcommand{\RegionMarkerTok}[1]{#1}
\newcommand{\InformationTok}[1]{\textcolor[rgb]{0.56,0.35,0.01}{\textbf{\textit{#1}}}}
\newcommand{\WarningTok}[1]{\textcolor[rgb]{0.56,0.35,0.01}{\textbf{\textit{#1}}}}
\newcommand{\AlertTok}[1]{\textcolor[rgb]{0.94,0.16,0.16}{#1}}
\newcommand{\ErrorTok}[1]{\textcolor[rgb]{0.64,0.00,0.00}{\textbf{#1}}}
\newcommand{\NormalTok}[1]{#1}
\usepackage{longtable,booktabs}
\usepackage{graphicx,grffile}
\makeatletter
\def\maxwidth{\ifdim\Gin@nat@width>\linewidth\linewidth\else\Gin@nat@width\fi}
\def\maxheight{\ifdim\Gin@nat@height>\textheight\textheight\else\Gin@nat@height\fi}
\makeatother
% Scale images if necessary, so that they will not overflow the page
% margins by default, and it is still possible to overwrite the defaults
% using explicit options in \includegraphics[width, height, ...]{}
\setkeys{Gin}{width=\maxwidth,height=\maxheight,keepaspectratio}
\IfFileExists{parskip.sty}{%
\usepackage{parskip}
}{% else
\setlength{\parindent}{0pt}
\setlength{\parskip}{6pt plus 2pt minus 1pt}
}
\setlength{\emergencystretch}{3em}  % prevent overfull lines
\providecommand{\tightlist}{%
  \setlength{\itemsep}{0pt}\setlength{\parskip}{0pt}}
\setcounter{secnumdepth}{5}
% Redefines (sub)paragraphs to behave more like sections
\ifx\paragraph\undefined\else
\let\oldparagraph\paragraph
\renewcommand{\paragraph}[1]{\oldparagraph{#1}\mbox{}}
\fi
\ifx\subparagraph\undefined\else
\let\oldsubparagraph\subparagraph
\renewcommand{\subparagraph}[1]{\oldsubparagraph{#1}\mbox{}}
\fi

%%% Use protect on footnotes to avoid problems with footnotes in titles
\let\rmarkdownfootnote\footnote%
\def\footnote{\protect\rmarkdownfootnote}

%%% Change title format to be more compact
\usepackage{titling}

% Create subtitle command for use in maketitle
\newcommand{\subtitle}[1]{
  \posttitle{
    \begin{center}\large#1\end{center}
    }
}

\setlength{\droptitle}{-2em}

  \title{Розеттский камень}
    \pretitle{\vspace{\droptitle}\centering\huge}
  \posttitle{\par}
    \author{Пуассон, фея и два мексиканских негодяя}
    \preauthor{\centering\large\emph}
  \postauthor{\par}
      \predate{\centering\large\emph}
  \postdate{\par}
    \date{2019-03-15}

\usepackage{booktabs}

\begin{document}
\maketitle

{
\setcounter{tocdepth}{1}
\tableofcontents
}
\chapter{Напутственное слово}\label{-}

\chapter{Установка софта}\label{installsoft}

\chapter{Коан о простой линейной регрессии}\label{simplereg}

Построим простую линейную регрессию в R.

Загрузим необходимые пакеты и импортируем данные.

\begin{Shaded}
\begin{Highlighting}[]
\KeywordTok{library}\NormalTok{(tidyverse) }\CommentTok{# для манипуляций с данными и построения графиков}
\KeywordTok{library}\NormalTok{(rio) }\CommentTok{# для чтения .dta файлов}
\KeywordTok{library}\NormalTok{(car) }\CommentTok{# для линейных гипотез}
\NormalTok{df =}\StringTok{ }\KeywordTok{import}\NormalTok{(}\DataTypeTok{file =} \StringTok{"us-return.dta"}\NormalTok{)}
\end{Highlighting}
\end{Shaded}

Исследуем наш датасет.

\begin{Shaded}
\begin{Highlighting}[]
\KeywordTok{head}\NormalTok{(df) }\CommentTok{# первые 6 наблюдений}
\NormalTok{df =}\StringTok{ }\KeywordTok{rename}\NormalTok{(df, }\DataTypeTok{n =}\NormalTok{ A, }\DataTypeTok{date =}\NormalTok{ B) }\CommentTok{# дадим столбцам осмысленные названия :)}
\CommentTok{# sum(is.na(df)) # проверим наличие пропусков skimr::skim}
\NormalTok{df =}\StringTok{ }\KeywordTok{na.omit}\NormalTok{(df) }\CommentTok{# и избавмся от них}
\end{Highlighting}
\end{Shaded}

Будем верить в CAPM :) Оценим параметры модели для компании MOTOR.
Соответсвенно, зависимая переменная - разница доходностей акций MOTOR и
безрискового актива, а регрессор - рыночная премия.

\begin{Shaded}
\begin{Highlighting}[]
\NormalTok{df <-}\StringTok{ }\KeywordTok{mutate}\NormalTok{(df, }\DataTypeTok{y =}\NormalTok{ MOTOR }\OperatorTok{-}\StringTok{ }\NormalTok{RKFREE, }\DataTypeTok{x =}\NormalTok{ MARKET }\OperatorTok{-}\StringTok{ }\NormalTok{RKFREE)}
\NormalTok{ols <-}\StringTok{ }\KeywordTok{lm}\NormalTok{(y }\OperatorTok{~}\StringTok{ }\NormalTok{x, }\DataTypeTok{data =}\NormalTok{ df)}
\KeywordTok{summary}\NormalTok{(ols)}
\end{Highlighting}
\end{Shaded}

\begin{verbatim}

Call:
lm(formula = y ~ x, data = df)

Residuals:
      Min        1Q    Median        3Q       Max 
-0.168421 -0.059381 -0.003399  0.061373  0.182991 

Coefficients:
            Estimate Std. Error t value Pr(>|t|)    
(Intercept) 0.005253   0.007200   0.730    0.467    
x           0.848150   0.104814   8.092 5.91e-13 ***
---
Signif. codes:  0 '***' 0.001 '**' 0.01 '*' 0.05 '.' 0.1 ' ' 1

Residual standard error: 0.07844 on 118 degrees of freedom
Multiple R-squared:  0.3569,    Adjusted R-squared:  0.3514 
F-statistic: 65.48 on 1 and 118 DF,  p-value: 5.913e-13
\end{verbatim}

Визуализируем зависимость регрессора и зависимой переменной на графике.

\begin{Shaded}
\begin{Highlighting}[]
\KeywordTok{ggplot}\NormalTok{(df, }\KeywordTok{aes}\NormalTok{(x, y)) }\OperatorTok{+}\StringTok{  }\KeywordTok{geom_point}\NormalTok{(}\DataTypeTok{shape=}\DecValTok{1}\NormalTok{) }\OperatorTok{+}\StringTok{ }
\StringTok{  }\KeywordTok{geom_smooth}\NormalTok{(}\DataTypeTok{method=}\NormalTok{lm)}
\end{Highlighting}
\end{Shaded}

\includegraphics{02-simplereg_files/figure-latex/plot-1.pdf}

\begin{Shaded}
\begin{Highlighting}[]
\KeywordTok{linearHypothesis}\NormalTok{(ols, }\KeywordTok{c}\NormalTok{(}\StringTok{"(Intercept) = 0"}\NormalTok{, }\StringTok{"x = 1"}\NormalTok{))}
\end{Highlighting}
\end{Shaded}

\begin{verbatim}
Linear hypothesis test

Hypothesis:
(Intercept) = 0
x = 1

Model 1: restricted model
Model 2: y ~ x

  Res.Df     RSS Df Sum of Sq      F Pr(>F)
1    120 0.74108                           
2    118 0.72608  2  0.014998 1.2187 0.2993
\end{verbatim}

\chapter{Модели бинарного выбора}\label{binchoice}

\chapter{Модели упорядоченного выбора}\label{ordchoice}

\chapter{Пуассоновская регрессия}\label{poisreg}

\chapter{Модели неупорядоченного выбора}\label{disordered}

\chapter{Интcтрументы для простой регрессии}\label{instruments}

\chapter{ARMA}\label{arma}

\chapter{Панельные данные}\label{paneldata}

\chapter{Гетероскедастичность в простой регрессии}\label{heterosked}

\chapter{PCA}\label{pca}

\chapter{Динамические панели}\label{dinpanel}

\chapter{TOBIT, HECKIT}\label{tobit_heckit}

\chapter{Treatment effect}\label{treatment}

\chapter{Что-то там про совместимость и языки}\label{compatability}

\chapter{Словарь}\label{dict}

\bibliography{book.bib,packages.bib}


\end{document}
